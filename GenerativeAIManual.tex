\documentclass[9pt, technote]{IEEEtran}
\IEEEoverridecommandlockouts

\usepackage{cite}
\usepackage{amsmath,amssymb,amsfonts}
\usepackage{algorithmic}
\usepackage{graphicx}
\usepackage{textcomp}
\usepackage{xcolor}
\usepackage{amsmath}
\usepackage{url}
\def\BibTeX{{\rm B\kern-.05em{\sc i\kern-.025em b}\kern-.08em
    T\kern-.1667em\lower.7ex\hbox{E}\kern-.125emX}}
\begin{document}

\title{Code Management and Generative AI Best Practices\\
    \large A Collaborative Review\\
    \large EE450 Military Robotic Applications

}

% When you contribute to this document, add an \author section for yourself
\author{\IEEEauthorblockN{Cadet 1}\\
\IEEEauthorblockA{\textit{United States Military Academy}\\
West Point, NY \\
First.Last@westpoint.edu}\\
\and
\IEEEauthorblockN{Cadet 2}\\
\IEEEauthorblockA{\textit{United States Military Academy}\\
West Point, NY \\
First.Last@westpoint.edu}
}
\IEEEauthorblockN{Cadet Taylor Brown}\\
\IEEEauthorblockA{\textit{United States Military Academy}\\
West Point, NY \\
Taylor.Brown@westpoint.edu}

\author{\IEEEauthorblockN{Cadet Charlotte Richman}\\
\IEEEauthorblockA{\textit{United States Military Academy}\\
West Point, NY \\
charlotte.richman@westpoint.edu}\\

\author{\IEEEauthorblockN{Cadet Meghan DeClue}\\
\IEEEauthorblockA{\textit{United States Military Academy}\\
West Point, NY \\
meghan.declue@westpoint.edu}\\
}

\maketitle

\begin{abstract}
This is a collection of best-practice proposals by the cadets of EE450 Military Robotic Applications.
The cadets use this document to share their perspectives, experiences, and recommendations on the use
of software IDEs (specifically, Visual Studio Code and the Arduino IDE), robotics project code structure,
and the use of generative artificial intelligence in completing robotics projects. This is a
"by-cadets-for-cadets" living record that both encourages reflection and guides future cadets in their
robotics endeavors.
\end{abstract}
\section{Instructions to Contributors}
This document is divided into four parts:
\begin{enumerate}
    \item Getting Started with Arduino Programming
    \item Code Management and Structure
    \item Integrating Generative AI
    \item Testimonials and Examples
\end{enumerate}

You may contribute to any or all of these sections as much or as little as you like. It is critical that
your contributions remain \textit{high quality} and \textit{easy to understand}. Do your best to place
your inputs in the correct sections and subsections, but feel free to create your own sections or subsections
if you think you need to. You must not, under any circumstances, provide direct answers to any of the course
projects, mini-projects, quizzes, or other assignments --- the purpose is to guide other cadets on their
own problem-solving, not to solve the problem for them.

Contributions to this document must be made via a \textit{pull request} to the appropriate GitHub repository.
This first requires you to \textit{clone} the project repository and create a new \textit{branch}. If you need
guidance on how GitHub works, or how to submit a pull request, you can check the provided references.\cite{github_branch} \cite{github_pull}
\section{Getting Started with Arduino Programming}
% Use this section to discuss how to set up the IDE, how to install libraries, and how to upload code to the Arduino microcontroller.
% Any generic guidance not specific to one IDE can go here, before the subsections.
\subsection{Visual Studio Code}

\subsection{Arduino IDE}

\subsubsection{Installing Libraries}
Libraries are a crucial component and starting point towards success in EE450. There are multiple ways to download libraries, which may be necessary to troubleshoot if you run into issues, however this will describe the most common two.

\paragraph{Library Manager}
\begin{enumerate}
    \item The Library Manager in Arduino IDE looks like stacked books on the left sidebar. Other ways to navigate here are to click \texttt{Tools} and then \texttt{Manage Libraries} or click on \texttt{Sketch}, \texttt{Include Library}, and then \texttt{Manage Libraries} to navigate to the Library Manager window.
    \item Within the Library Manager, you will see a search box. Typing in this search box will search all installed and available libraries within Arduino IDE.
    \item Search for the library you need like Pixy2, Servo.h, etc. Once selected click ``Install,'' allow the system to process and show that the library is installed.
    \begin{figure}[h!]
\centering
\begin{minipage}{0.32\textwidth}
    \centering
    \includegraphics[width=\linewidth]{https://raw.githubusercontent.com/taylorbrown917-commits/EE450_Best_Practices/7e23eb898bca26e58225763291beca5383b34c06/Screenshot%202025-12-02%20143227.png}
\end{minipage}
\hfill
\begin{minipage}{0.32\textwidth}
    \centering
    \includegraphics[width=\linewidth]{https://raw.githubusercontent.com/taylorbrown917-commits/EE450_Best_Practices/7e23eb898bca26e58225763291beca5383b34c06/Screenshot%202025-12-06%20202003.png}
\end{minipage}
\hfill
\begin{minipage}{0.32\textwidth}
    \centering
    \includegraphics[width=\linewidth]{https://raw.githubusercontent.com/taylorbrown917-commits/EE450_Best_Practices/7e23eb898bca26e58225763291beca5383b34c06/Screenshot%202025-12-06%20202012.png}
\end{minipage}

\caption{EE450 Best Practices Images}
\end{figure}
\end{enumerate}

\paragraph{\texttt{.zip} Library}
Sometimes during the class you will come across a library that is not included in Arduino IDE and will require you to source the \texttt{.zip} file. Here are the easiest steps:
\begin{enumerate}
    \item Download the non-extracted version of the \texttt{.zip} file.
    \item In Arduino IDE go to \texttt{Sketch}, then \texttt{Include Library}, and add the \texttt{.zip} Library.
    \item Select the \texttt{.zip} library and open it. Wait for the program to confirm the library has been installed.
    \begin{figure}[h!]
\centering
\includegraphics[width=0.45\textwidth]{https://raw.githubusercontent.com/taylorbrown917-commits/EE450_Best_Practices/7e23eb898bca26e58225763291beca5383b34c06/Screenshot%202025-12-06%20202052.png}
\caption{Additional EE450 Best Practices Image}
\end{figure}
\end{enumerate}

\textbf{Note:} Some libraries might need some troubleshooting. Extracting all files from the \texttt{.zip} and saving them in a folder on your computer, and then moving the library into the Arduino \texttt{libraries} folder on your computer, is a way to resolve this issue and try again.


\section{Code Management and Structure}
% Below are a few recommended subsections for you to contribute to, but feel free to create your own if you think it is appropriate.
\subsection{Incorporating the Sense-Decide-Act Paradigm} 
% Did you structure your programs this way? Was it successful or not?

\subsection{Implementing States and Using State Diagrams} 
% Are state diagrams helpful at the start of a project? How did you use them?
I highly recommend creating a state diagram and a block diagram at the start of every project. In my experience in EE450, state diagrams 
and block diagrams were some of the most useful tools in helping me complete each project.
The state diagrams provided a clearly layed out visual representation of what the robot is supposed to be able to do, and how it
should transition from one behavior or state to another. By mapping out each state, the conditions for each transition, and 
the actions associated with each state, I was able to generate my code and final product much more efficiently. 
State diagrams are extremely helpful at the start of EE450 projects even if they are not required/graded because they give you a clear visual of how your robot should behave. When you breakdown your system into states, it becomes much easier to understand what the robot is supposed to do at any given time and what events cause it to switch behaviors.

In EE450, your robot will almost always need to move between different modes (for example: \texttt{SAFE}, \texttt{FOLLOW}, \texttt{OBJECT DETECTED}, or \texttt{ATTACK/DEFEND}). A state diagram lets you map these different mdoes out before you begin to write any code.

For this project with these states, we used a state diagram to plan the overall flow of the robot. Before touching the Arduino code, we mapped out:
\begin{itemize}
    \item what each state is supposed to do,
    \item what inputs trigger a transition (like a button press or sensor reading),
    \item and what the robot should do after switching states.
\end{itemize}

Having the state diagram made implementation much easier because the code could then be built one state at a time. Each state became its own section of the \texttt{loop()} function or its own block. If something was not functioning properly, it was easier to trace which state caused the issue. Especially when working with a new platform like the Traxxas, differentiating between hardware and software issues was eased by using a state machine diagram. Also, if it is demo day and you do not have a fully working robot, having individual states from your state diagram functioning can still earn you points. Breaking your code into clear states allows you to get portions of the robot working even if the whole system is not complete. This means you can demonstrate specific state behaviors and still receive partial credit instead of ending up with no points if the full integration is giving you trouble.

Overall, state diagrams can help you:
\begin{itemize}
    \item understand the robot's behavior before coding or asking chatgpt to help you build your code,
    \item write cleaner, more concise, and organized code,
    \item help you understand chatgpt generated code more efficiently, 
    \item and troubleshoot faster when things don’t work.
\end{itemize}

They are a simple but an extremely useful tool, especially when your robot starts getting more complex. For some, in the earlier projects you may feel like the state diagram is unnecessary, however highly reccomend getting some repetitions early on, because they are extremely helpful in later projects. Others may beenfit from a state diagram from the start of Project 2 to the conclusion of Project 5. 

The block diagrmas are 
helpful for visualizing the overall system, including sensors, control logic, servos, buzzers, LEDs, and other components.
These diagrams allowed me to break down complex behaviors into manageable parts, making it easier to implement and test each component 
individually before integrating them into the final system.

\subsection{File Structure and Version Management --- Avoid Drowning in Code} 
% Arduino sketches can grow into large, complex files. How did you organize them internally so you are always confident
% that your program is doing what you want?
To help manage me code and avoid getting lost in what I was doing, I made sure to implement a lot of comments throughout 
my code. This helped me to remember what each line and section of code was executing and why the code was structred as it was.
FOr example, if I was using multiple sensors within the same project, I would make a header comment at the top of the functions I 
called for each individual sensor so that the commands did not get confused and merged together, leading to error messages
and bugs. Overall, comments helped me stay organized, helped me understand coding better, and made it easier to debug and correct
mistakes as I went along in each project. 

\section{Integrating Generative AI}

\subsection{Registering for Copilot Pro} 
% provide instructions for registering for Copilot Pro using a student license


\subsection{Managing and Integrating Generated Code} 
% give guidance on how to prompt Copilot -- how do you structure the prompt, what context do you give it, and what do you
% do with what it generates? 
Using generated code from sources like ChatGPT, Copilot, etc., can be an extremely useful tool in working as a teammate to assist in fulfilling the project requirements and aiding in understanding the prompt. However, actually getting your robot to fully perform the task requirements will require you to understand the code. It is best to use AI to accelerate the progress, but not replace your understanding of solving the problem. Here are some recommended tips for prompting Copilot and other artificial intelligence platforms and managing the outputs:

\begin{enumerate}
    \item Copilot is most effective when it knows exactly what you would like it to help you with. Think of components like: What does this code do? What variables are being considered as inputs, and what are the outputs? What are the specifications and constraints of the project, including what hardware is being used in the specific project? How should the states transition?  
    Here is an example of a prompt:
    \begin{quote}
    ``Build code for the DRIVE state that reads ultrasonic sensor distance and drives the servos forward only when distance is less than 30. Print sensor values to the serial monitor.''
    \end{quote}
    \item Give Copilot the current state of your project. This means: What components have you already constructed in the process? By giving Copilot the pin mapping, defined variables, and completed state diagram, the code it generates will be easier to integrate into your own code.
    \item What do you do with what it generates? Make sure you read it carefully and ensure there are no functions or variables that conflict with your existing code. Change some components of it, like names or parameters, so that it fits with your intended design. Test the code incrementally or by itself first, this means either test the code in a new sketch or add serial prints throughout the code to ensure the behavior is congruent with your intended design. Recognize that the produced code will not be perfect, but it will help show you different methods or structures of completing tasks.
\end{enumerate}

Generated code can be very helpful in completing the more complicated aspects of a project and it helps the you to save time in long projects.
However, it is critical that you understand how the code is being implemented in your project to ensure that it is functioning how you want it to. 
One way to do this is to create an outline for your code first.
Before prompting generative AI to write code for you, it is important to first create a test plan for your components and make sure that each piece properly works.
For example, if you are using a distance sensor to detect objects, you should start with short test code to make sure the sensor works.
This is a great way to use AI to help you save team through the component testing process.
You can ask ChatGPT for a short sketch to test the function of each component and individually test them. 
Once you know that all of the components work, you can start to piece together your final coding sketch. 
Start by assigning each component in your robot to a pin and defining any variables you need. Make sure you include the proper libraries.
Next, define the functions you want to include in your code that you implement from your state diagram. 
Once you have your structure laid out, you can start to prompt generative AI to help you fill in the functions you need. 
Generative AI is very helpful for helping you to structure your functions to work how you want.
Make sure to provide AI with context on how you plan on using the function and how it relates to the other functions in your code.
Lastly, use generative AI to help you debug your code. It can provide helpful explanations for why errors are occurring and provide guidance for how to fix them. 



\subsection{Documentation and Citation of Generated Code} 
% How do you document and cite generated code? How do you avoid plagiarism and ensure all generated code is clearly marked?
When using generative AI to assist you in writing code for this class, it is essential that you properly cite the sources that you used for two primary reasons.
The first, and most important, is maintaing academic integrity in your work. 
The second is to provide a record for yourself to understand what parts of your code were generated from AI and what sections were written by you.
Oftentimes you will be able to recycle code from previous projects and modify it to meet the criteria of the project you are currently working on. 
By ensuring that you cite each section of code, it helps you to understand how the code works and how you can use it to help you better understand coding logic and implementation.
It is crucial you do two things to properly document your code.
The first thing you will do is create the citation you will include in your references page. 
You will create a citation in accordance with the DAAW that includes what model of AI you used (for example: ChatGPT 5.0), the prompt you inputted into that chatBot, and then an explanation of how you implemented the assistance into your code. 
The second thing you wil do is comment in your code exactly which lines were generated by AI.
An example of this would be the following: assistance to the editor from ChatGPT 5.0, commented next to the line of code. 


\section{Testimonials and Examples}
Feel free to add any other guidance, examples (good or bad!), or advice to other cadets here.

\subsection{Advice from CDT Charlotte Richman}
I highly recommend completing a test integration plan at the onset of each project. This plan should
outline how you are going to go about testing each individual componet of your robot, to include servos, sensors, LEDs, etc., 
and then how you will integrate all those seperate components into one robot that functions as you want it to. 
This plan will save you a lot of time and frustration when you get into more complex projects and problems with many 
moving parts because you will know exactly what is and is not functioning properly before you are too deep in and 
have to backtrack, sometimes all the way back to the beginning. Having a clear plan for testing and integration will save you
a lot of time and evergy in the long run, and it will help you to understand what each component of your robot is 
doing, as well as how it is contributing to your entire system. I think that is pretty cool! 

\subsection{Advice from CDT Meghan DeClue}
If I had to give one piece of advice, it would be to start with your state diagram for the project. Properly understanding \textit{then}
the states that you want your robot to be in and how you want them to transition will make the coding process much easier. 
From there, I would recommend breaking down each state into a function embedded in your loop() function. This will help you to keep your \textit{then}
code organized and easy to understand. By only keeping functions in your loop(), you can simplify your code and make it easier to read and debug \textit{then}
in the case of having any errors. Complex projects tend to have hundreds of line of codes and can become hard to understand very quickly. \textit{then}
Keeping your code organized by state and function will help to avoid confusion and it will make coding your project more efficient.

\subsection{Advice from CDT X} % example
If I had to give one piece of advice, it's to research the sensors I'm using and understand how they work, \textit{then}
build my state diagram and think about what I actually want the robot to do in terms of sensors and actuators, \textit{then}
think about how I want to prompt the generative AI. Starting by inputting the problem statement into Copilot never worked
for me a single time --- it always gave me something I didn't understand and had no idea how to fix, causing problems later.

\bibliographystyle{plain}
\bibliography{References}


\end{document}
